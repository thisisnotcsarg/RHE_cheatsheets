\documentclass{article}

\usepackage[a4paper, landscape, margin=5pt]{geometry}
\usepackage{hyperref}       % links
\usepackage{multicol}       % for columns layout
\usepackage{blindtext}      % lorem ipsum
\usepackage{listings}       % code
\usepackage{xcolor}          % color

% settings
\setlength{\parindent}{0pt}     % no first row paragraph indentation
\setlength{\parskip}{0pt}       % Space between paragraphs
\setlength{\multicolsep}{0pt}   % no bottom and top space in multicols
\setlength{\columnsep}{2pt}     % column separation
\pagestyle{empty}               % no page numbers
% setting listings to Racket style
\lstdefinestyle{Racket}{
    aboveskip=0pt,
    belowskip=0pt,
    commentstyle=\color{orange},
    basicstyle=\ttfamily\color{darkgray},
    keywordstyle=\ttfamily\bfseries\color{black},
    stringstyle=\color{blue},       
    breaklines=true,                              
    keepspaces=true,
    language=Lisp,
    mathescape=true,
    alsoletter={!, ?},
    keywords={define, let, set!, vector, vector-set!, apply, null?, list?, equal?, define-syntax, syntax-rules, map, string?, zero?, foldr, foldl, split-at, take, drop, let-values, struct, if, where, cond, s?, set-s-var!, begin, for-each, pair?, unless}
}
\lstset{style=Racket} 

\begin{document}
\textbf{Racket CheatSheet} - \href{https://github.com/isthissuperuser}{https://github.com/isthissuperuser} \textbar{ }  \textbf{legenda}: \{ \} optional, [ ] one or more, 
\begin{multicols*}{3}

\hrulefill

\textbf{MISC}
\begin{lstlisting}
(define var val)
(let {name-let} ([var val]) 
    (body)
    ...
    {name-let (var val)})
(set! name value)
(begin
  body
  ...)
(apply f lst)
(null? lst) ; is empty list
(pair? lst) ; (not (null? lst))
(list? lst) ; is list
(equal? var val)
(not cond) ; negation
(string? var)
(zero? n) ; n == 0
(unless cond)
    body ; false
(error "error") ; display error
\end{lstlisting}

\hrulefill

\textbf{CONTROL FLOW}
\begin{lstlisting}
(if c
  true
  false)
(cond 
  [(c1) (body)])
(when (c) ; as if but just true
  (begin
    ...))
\end{lstlisting}

\hrulefill

\textbf{VECTORS}
\begin{lstlisting}
#(1 2 3) ; immutable
(vector 1 2 3) ; mutable
(vector-set! v 1 0) ; v[1] = 0
(make-vector length {default_v}) ; mutable
(build-vector length f(index))
(vector-length v) ; returns size
(vector-ref v i) ; return v[i]
(for ([i v]) (body)) ; cycle
\end{lstlisting}

\hrulefill

\textbf{LISTS}
\begin{lstlisting}
(list) ; create empty list 
(define lst (list a b))
(define lst '(a b))
(car lst) ; take head
(cdr lst) ; take rest
(a . '()) ; concat elems
(map (f) lst)
; starts from the beginnig
; space complexity: O(1)
; tail-recursive
(foldl f first_elem lst)
; starts from the end
; space complexity: O(n)
; no tail-recursive
(foldr f first_elem lst)
(let-values ([(var1 var2) (split-at lst idx)])
  (body)); split at idx into var1, var2
(take lst n) ; ret first n elem
(drop lst n) ; ret elems after n
(for-each f(elem) lst)
(for ([i (in-list l)])
    (body))
\end{lstlisting}

\hrulefill

\textbf{FUNCTIONS}
\begin{lstlisting}
(define f $\lambda$((arg . args)(body)))
(define (f arg . args)
  (body))
(define (f n acc);tail recursive
  (if (base_case)
    acc
    (f (step n) (iteration acc n))))
(define (f) ; closure
  (let ((var val))
    ($\lambda$ () ; state preserved
      (inc1 var)))) 
\end{lstlisting}

\hrulefill

\textbf{STRUCTS}
\begin{lstlisting}
(struct s
  (var1 ; immutable field
  [var2 #:mutable]))
(s? var) ; is name
(set-s-var! s val)
\end{lstlisting}

\hrulefill

\textbf{MACROS}
\begin{lstlisting}
(define-syntax name
  (syntax-rules () ; decorative
    [(syntax) ; first case
     (body)] ; 1 stmt here
    [(_ ...) ()] ; catch all
))
\end{lstlisting}

\hrulefill

\textbf{CONTINUATIONS}
\begin{lstlisting}
; calling k substitutes call/cc block with arg and jumps
(let ([cc 
  (call/cc (lambda (k)
    (k k)))])
  (cc cc)) ; same as calling k
\end{lstlisting}

\hrulefill

\textbf{PATTERNS}
\begin{lstlisting}
; call/cc list push/pop
; choice puts in stack checkpoint that call choice again but with all the other choices and returns a choice. Function fail fails if no other choices are present otherwise it pops and call from the stack
(define (unknown-method ls)
  (error "Unknown method" (car ls)))
(define-syntax define-dispatcher
  (syntax-rules (methods: parent:)
    ((_ methods: (mt ...) parent: p)
     (lambda (message . args)
       (case message
         ((mt) (apply mt args))
         ...
         (else (apply p (cons message args))))))
    ((_ methods: mts)
     (define-dispatcher methods: mts parent: unknown-method))))
\end{lstlisting}

\hrulefill

\textbf{TIPS}
\begin{lstlisting}
; Quando crei sintassi la prima parola che da inizo a tutto. Le altre parole sono decorative. Puoi usare _ al posto della parola master della sintassi. Le parentesi se ti servono devi metterle. I ... inheritano automaticamente le ripetizioni di certe parti
; dentro le parentesi gia si esegue a meno che non siano array!
; una lamda dentro a una funzione e una closure e la funzione preserva lo stato
; ... devono essere usati per stmt vicini
\end{lstlisting}

\end{multicols*}
\end{document}
