\documentclass{article}

\usepackage[a4paper, landscape, margin=5pt]{geometry}
\usepackage{hyperref}       % links
\usepackage{multicol}       % for columns layout
\usepackage{blindtext}      % lorem ipsum
\usepackage{listings}       % code
\usepackage{xcolor}          % color

% settings
\setlength{\parindent}{0pt}     % no first row paragraph indentation
\setlength{\parskip}{0pt}       % Space between paragraphs
\setlength{\multicolsep}{0pt}   % no bottom and top space in multicols
\setlength{\columnsep}{2pt}     % column separation
\pagestyle{empty}               % no page numbers
% setting listings to Racket style
\lstdefinestyle{Haskell}{
    aboveskip=0pt,
    belowskip=0pt,
    commentstyle=\color{orange},
    basicstyle=\ttfamily\color{darkgray},
    keywordstyle=\ttfamily\bfseries\color{black},
    stringstyle=\color{blue},       
    breaklines=true,                              
    keepspaces=true,
    language=Haskell,
    mathescape=true,
    alsoletter={!, ?},
    keywords={data, map, show, module, let, in, zip, where, class, false, instance, take, not, deriving, id, Monad, >>=, return, foldr, if, then, else, maximum}
}
\lstset{style=Haskell}
\newcommand{\dollar}{\mbox{\textdollar}}

\begin{document}
\textbf{Haskell CheatSheet} - \href{https://github.com/isthissuperuser}{https://github.com/isthissuperuser} \textbar{ }  \textbf{legenda}: \{ \} optional, [ ] one or more, 
\begin{multicols*}{4}

\hrulefill

\textbf{MISC}
\begin{lstlisting}
3
data Tree2 a = Nil | B a (Tree2 a) (Tree2 a)
data Gtree a = Nil | Gtree a [Gtree a] deriving (Show)
data Name a = Name {getterX, getterY :: a} 
 -- constructor takes 2 arg of different type a and also expose the getters
let var1 = value1
    var2 = value2
in  body
$\dollar$ -- evaluate before right expression
show -- print
v@(T x' l' c' r') -- pick up data on the right inside v
if condition then body else body
f (x, y) = x -- couple as arg
\end{lstlisting}

\hrulefill

\textbf{LISTS}
\begin{lstlisting}
[a] -- list of variable type
list !! 0 -- take element at index 0
[1, 3 .. 10] -- [1, 3, 6, 9]
1:[] -- cons operator: single:list not contrary
++ -- list concat
take 3 list -- return first 3 elems
drop 3 list -- remove first 3 elems and return new list
head list -- return first element
tail list -- return list without first element
last l -- return last elem from list
init l -- returns all elem of l except last 
reverse list -- reverse list
lst [ x * 2 | x <- [0, 1 ..]] -- even numbers
zip lst1 lst2 -- couples from each lst
null v -- returns true if v empty
length l -- returns the length of the list
\end{lstlisting}

\hrulefill

\textbf{FUNCTIONS}
\begin{lstlisting}
fname arg = body
map (f1 . f2) lst -- first map f2 then f1
concatMap f lst -- for every elem f creates a list, concatMap concats all the lists into a single list
(\ x -> x * 2) -- lambda
fname arg@(cur:next:rest) = body -- arg is list decomposed
fname num
 | f1 < 2 = "small"
 | f1 > 2 = "big"
 | otherwise = "normal"
 where
    f1 = fbody
id -- identity function
4 `div` 2 -- integer division
filter f v -- f boolean function, returns a list of the elements of v for which f returned true
maximum v--returns v max element
\end{lstlisting}

\hrulefill

\textbf{CLASSES}
\begin{lstlisting}
class Equal a where -- pseudo class
    (==) :: a -> a -> bool
    x /= y = not (x == y)
instance (Equal a) => Equal (Tree2 a) where
    Nil == s = s == Nil
    s == Nil = s == Nil
    (B v l r) == (B v2 l2 r2) = v == v2 && l == l2 && r == r2
    _ == _ = False
-- instantiating Equal pseudo class with a Tree as arg
\end{lstlisting}

\hrulefill

\textbf{FOLDABLE}
\begin{lstlisting}
instance Foldable Slist where
  foldr f acc (Slist list len) = foldr f acc list
instance Foldable Tree where
    foldr :: (a -> b -> b) -> b -> Tree a -> b
    foldr f acc Nil = acc
    foldr f acc (Leaf x) = f acc x
    foldr f acc (B l r) = foldr f (foldr f acc r) l
\end{lstlisting}

\hrulefill

\textbf{FUNCTORS}
\begin{lstlisting}
instance Functor Slist where
  fmap :: (a -> b) -> Slist a -> Slist b
  fmap f (Slist list len) = Slist (fmap f list) len
instance Functor Tree where
  fmap :: (a -> b) -> Tree a -> Tree b
  fmap f Nil = Nil
  fmap f (Leaf a) = Leaf (f a)
  fmap f (B l r) = B (fmap f l) (fmap f r)
\end{lstlisting}

\hrulefill

\textbf{APPLICATIVES}
\begin{lstlisting}
--[(+1), (*2), (^3)] <*> [1,2,3]
--[2, 3, 4, 2, 4, 6, 1, 8, 27]
-- partial f applied to list and concat
instance Applicative Slist where
  pure :: a -> Slist a
  pure a = Slist [a] 1
  (<*>) :: Slist (a -> b) -> Slist a -> Slist b
  (Slist flist _) <*>  (Slist a alen) = Slist (flist <*> a) alen
instance Applicative Tree where
  pure x = Leaf x
  Nil <*> _ = Nil
  _ <*> Nil = Nil
  (Leaf f) <*> (Leaf x) = Leaf (f x)
  (Leaf f) <*> (B left right) = B (Leaf f <*> left) (Leaf f <*> right)
  (B leftF rightF) <*> tree = B (leftF <*> tree) (rightF <*> tree)
instance Applicative Tree where
  pure :: a -> Tree a
  pure = Leaf
  (<*>) :: Tree (a -> b) -> Tree a -> Tree b
  Nil <*> _ = Nil
  _ <*> Nil = Nil
  (Leaf f) <*> t = fmap f t
  (B t1 t2) <*> t = B (t1 <*> t) (t2 <*> t)
ltconcat :: BTT (BTT a) -> BTT a
ltconcat t = foldr (<++>) Nil t
ltconcmap :: ((a->b) -> BTT b) -> BTT (a -> b) -> BTT b
ltconcmap f t = ltconcat (fmap f t)
instance Applicative BTT where
  pure :: a -> BTT a
  pure x = B2 x Nil Nil
  (<*>) :: BTT (a -> b) -> BTT a -> BTT b
  x <*> y = ltconcmap (\f -> fmap f y) x
(+++) :: Gtree a -> Gtree a -> Gtree a
Nil +++ s = s
s +++ Nil = s
(Gtree a l) +++ g = Gtree a (g:l)
instance Applicative Gtree where
  pure a = Gtree a []
  x <*> y = foldr (+++) Nil (fmap (\f -> fmap f y) x)
\end{lstlisting}

\hrulefill

\textbf{MONADS}
\begin{lstlisting}
instance Monad Result where
    Ok x >>= f = f x
    Err >>= _ = Err
instance Monad Slist where
  (>>=) :: Slist a -> (a -> Slist b) -> Slist b
  (Slist list len) >>= f = let finalL = (list >>= (\x -> let Slist xs _ = f x in xs)) in Slist finalL $\dollar$ length finalL
instance Monad Tree where
  (>>=) :: Tree a -> (a -> Tree b) -> Tree b
  Nil >>= f = Nil
  (Leaf a) >>= f = f a
  (B t1 t2) >>= f = B (t1 >>= f) (t2 >>= f)
\end{lstlisting}

\hrulefill

\textbf{STATE MONADS}
\begin{lstlisting}
import Control.Monad.State
type Stack = [Int]
popM :: State Stack Int
popM = state $\dollar$ \(x : xs) -> (x, xs)
pushM :: Int -> State Stack ()
pushM a = state $\dollar$ \xs -> ((), a : xs)
stackManipM :: State Stack Int
stackManipM = do
  pushM 3
  a <- popM
  popM
state0 = [1,2,3,4,5]
result = runState stackManipM state0
\end{lstlisting}

\hrulefill

\textbf{Examples}
\begin{lstlisting}
data Btree a = Leaf a | Branch (Btree a)(Btree a) deriving (Show, Eq)
instance Functor Btree where
    fmap f (Leaf x) = Leaf (f x)
    fmap f (Branch x y) = Branch (fmap f x) (fmap f y)
addLevel :: Btree a -> Btree a
addLevel t = Branch t t
btrees :: a -> [(Btree a)]
btrees x = (Leaf x) : [addLevel t | t <- btrees x]
incBtrees :: [Btree Integer]
incBtrees = (Leaf 1) : [ addLevel (fmap (+1) t) | t <- incBtrees]
counts :: [Integer]
counts = map (\x -> 2^x - 1) [1..]
\end{lstlisting}
\hrulefill
\begin{lstlisting}

\end{lstlisting}

\end{multicols*}
\end{document}
