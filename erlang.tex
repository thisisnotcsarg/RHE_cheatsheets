\documentclass{article}

\usepackage[a4paper, landscape, margin=5pt]{geometry}
\usepackage{hyperref}       % links
\usepackage{multicol}       % for columns layout
\usepackage{blindtext}      % lorem ipsum
\usepackage{listings}       % code
\usepackage{xcolor}          % color

% settings
\setlength{\parindent}{0pt}     % no first row paragraph indentation
\setlength{\parskip}{0pt}       % Space between paragraphs
\setlength{\multicolsep}{0pt}   % no bottom and top space in multicols
\setlength{\columnsep}{2pt}     % column separation
\pagestyle{empty}               % no page numbers
% setting listings to Erlang style
\lstdefinestyle{Erlang}{
    aboveskip=0pt,
    belowskip=0pt,
    commentstyle=\color{orange},
    basicstyle=\ttfamily\color{darkgray},
    keywordstyle=\ttfamily\bfseries\color{black},
    stringstyle=\color{blue},       
    breaklines=true,                              
    keepspaces=true,
    language=Erlang,
    mathescape=true,
    alsoletter={!, ?},
    keywords={}
}
\lstset{style=Erlang} 

\begin{document}
\textbf{Erlang CheatSheet} - \href{https://github.com/isthissuperuser}{https://github.com/isthissuperuser} \textbar{ }  \textbf{legenda}: \{ \} optional, [ ] one or more, 
\begin{multicols*}{3}

\hrulefill

\textbf{MISC}
\begin{lstlisting}
io:fwrite("Hello ~s~n", [Name])
% 1 == 1.0  % => True
% 1 =:= 1.0 % => False
RPS = [rock, paper, scissor] % atoms
% every -> ends with a ;
% if it is last arrow ends with a .
% statements inside same -> ends with,
\end{lstlisting}

\hrulefill

\textbf{CONTROL FLOW}
\begin{lstlisting}
start() ->
stop() ->
if
    ture_cond ->
        code;
    false_cond ->
        code
end,
case expr of
    val1 -> code;
    val2 -> code
end,
\end{lstlisting}

\hrulefill

\textbf{LISTS}
\begin{lstlisting}
emptyl = [].
['1'] % list containing the char 1
[x | xs] % cur and rest of list
[] ++ [] % list concat 
lists:member(soccer, Sports) % true if soccer atom is inside Sports list
lists:nth(rand:uniform(2), RPS) % extract the ith element from RPS, in this case a random index
lists:foreach(fun, list)
\end{lstlisting}

\hrulefill

\textbf{MAPS}
\begin{lstlisting}
Map = #{key => 0}. % initialization
maps:is_key(Key, Map) % check key
maps:get(Key, Map) % Return Val
maps:put(Key, Value, Map) % update
maps:find(Key, Map) % returns {ok, Value} | error
maps:values(Map) % List values 
maps:keys(Map) % list of keys
\end{lstlisting}

\hrulefill

\textbf{FUNCTIONS}
\begin{lstlisting}
greet(Name) when is_list(Name) ->
    "Hello " ++ Name ++ "!";
greet(_) ->
    "Invalid name".
\end{lstlisting}

\hrulefill

\textbf{PARALLELISM}
\begin{lstlisting}
Pid = spawn_link(fun() -> rps_loop() end) % spawn a process
spawn(?MODULE, btrees_body, [{leaf, N}, Pid]) % spawned with defined function and given args
Pid ! {self(), rock} % send message to process
receive
    {PlayerPid, PlayerChoice} ->
        PlayerPid ! rps_win(PlayerChoice, CPUChoice)
    after 3000 ->
        exit(ok)
end.
Self = self() % take your pid
register(Name, Pid)
unregister(Name)
\end{lstlisting}

\hrulefill

\textbf{EXAMPLES}
\begin{lstlisting}
deeprev([X | Xs]) -> deeprev(Xs) ++ [deeprev(X)];
deeprev(X) -> X.
deeprevp(L) ->
    Me = self(),
    drp(Me, L),
    receive
        {Me, R} -> R
    end.
drp(F, [P|N]) ->
    Me = self(),
    C1 = spawn(fun() -> drp(Me, P) end),
    C2 = spawn(fun() -> drp(Me, N) end),
    receive
        {C1, RP} ->
            receive
                {C2, RN} -> F ! {Me, RN ++ [RP]}
            end
    end;
drp(F, L) ->
    F ! {self(), L}.
% For NDPDA, register the states, Every state is a subprocess, THey receive messages and send messages to other states, Message is composed by: The whole string, the current character of the string and next characters and the current symbol in the stack and the next symbols, based on that it updates the states that is stored on the messages and routes to different states
\end{lstlisting}
\hrulefill
\begin{lstlisting}
inctree({leaf, X}) ->
    {leaf, X+1};
inctree({branch, X, Y}) ->
    {branch, inctree(X), inctree(Y)}.
incbtrees(F, T) ->
    receive
        next ->
            T1 = inctree(T),
            T2 = {branch, T1, T1},
            F ! T2,
            incbtrees(F, T2);
        stop -> T
    end.
\end{lstlisting}
\hrulefill
\begin{lstlisting}

\end{lstlisting}
\end{multicols*}
\end{document}
